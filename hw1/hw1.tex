%&"../mdl"
\begin{document}
    \title{作业一:MLQP}
    \maketitle

    \section{推导}
    \begin{problem}
        Suppose the output of each neuron in a multi-
layer perceptron is:

\begin{equation}\label{eq:def}
    x_{kj} = f\left(\sum_{i=1}^{N_{k-1}}(u_{kji}x_{k-1,i}^2+v_{kji}x_{k-1,i})+b_{kj}\right)
\end{equation}

where both $u_{kji}$ and $v_{kji}$ are the weights connecting the $i^\text{th}$ unit in the layer $k-1$ to the $j^\text{th}$ unit in the layer $k$, $b_{kj}$ is 
the bias of the $j^\text{th}$ unit in the layer $k$, $N_k$ is the number of 
units  if $1\leq k\leq M$ and $f(\cdot)$ is the sigmoidal activation 
function. 

Please derive a back-propagation algorithm for multilayer 
quadratic perceptron (MLQP) in on-line or sequential 
mode.
    \end{problem}

    \begin{figure}[h]
        \centering
        \includegraphics[height=0.25\textheight]{mlqp}
        \caption{MLQP}\label{fig:mlqp}
    \end{figure}

    \subsection{输出神经元}
        对于输出神经元信号 $x_{Mj}$ 来说,令 $d_{Mj}$ 为目标值,则对应的误差为
        \begin{equation*}
            e_{Mj} = d_{Mj} - x_{Mj}
        \end{equation*}
        使用均方误差计算损失函数
        \begin{equation}\label{eq:loss}
            \mathcal{E} = \sum_{j=1}^{N_{M}}\frac{1}{2}e_{Mj}^2
        \end{equation}
        则对于 $\mathit{net}_{Mj}$ 的梯度为
        \begin{align*}
            \frac{\partial\mathcal{E}}{\partial net_{Mj}} &= \frac{\partial\mathcal{E}}{\partial e_{Mj}}\frac{\partial e_{Mj}}{\partial x_{Mj}}\frac{\partial x_{Mj}}{\partial \mathit{net}_{Mj}} \\
            &= -e_{Mj}f^\prime(\mathit{net}_{Mj})
        \end{align*}
        定义对应的局部梯度(local gradient)为
        \begin{equation}
            \delta_{Mj} = -\frac{\partial\mathcal{E}}{\partial net_{Mj}} = e_{Mj}f^\prime(\mathit{net}_{Mj})
        \end{equation}
    \subsection{隐藏层}
    对于隐藏神经元上的 $net_{k-1,j}$ 而言,假设后一层反向传播来的局部梯度已知:
    \begin{equation*}
        \delta_{kj} = -\frac{\partial\mathcal{E}}{\partial\mathit{net}_{kj}}
    \end{equation*}
    则该层的局部梯度为
    \begin{align}
        \delta_{k-1,i} = -\frac{\partial\mathcal{E}}{\partial\mathit{net}_{k-1,i}} &= -\frac{\partial\mathcal{E}}{\partial x_{k-1,i}}\frac{\partial x_{k-1,i}}{\partial\mathit{net}_{k-1,i}} \nonumber\\
        &= -f^\prime(\mathit{net}_{k-1,i})\sum_{j=1}^{N_j}\frac{\partial\mathcal{E}}{\partial\mathit{net}_{kj}}\frac{\partial\mathit{net}_{kj}}{\partial x_{k-1,i}} \nonumber\\
        &= -f^\prime(\mathit{net}_{k-1,i})\sum_{j=1}^{N_j}\delta_{kj}(2u_{kji}x_{k-1,i}+v_{kji})
    \end{align}
    \subsection{更新权值}
    而根据公式 \eqref{eq:def},
    \begin{align*}
        \frac{\partial\mathit{net}_{kj}}{\partial u_{kji}} &= x_{k-1,i}^2 \\
        \frac{\partial\mathit{net}_{kj}}{\partial v_{kji}} &= x_{k-1,i}
    \end{align*}
    设定学习率分别为 $\eta_1,\eta_2$,则对应权重的修正值
    \begin{align*}
        \Delta u_{kji} = -\eta_1\frac{\partial\mathcal{E}}{\partial u_{kji}} 
        &=-\eta_1\frac{\partial\mathcal{E}}{\partial\mathit{net}_{kj}}\frac{\partial\mathit{net}_{kj}}{\partial u_{kji}}=\eta_1\delta_{kj}x_{k-1,i}^2 \\
        \Delta v_{kji} = -\eta_2\frac{\partial\mathcal{E}}{\partial v_{kji}} &=-\eta_1\frac{\partial\mathcal{E}}{\partial\mathit{net}_{kj}}\frac{\partial\mathit{net}_{kj}}{\partial v_{kji}}= \eta_2\delta_{kj}x_{k-1,i}
    \end{align*}
    为了提高学习速率,引入动量
    \begin{align}
        \Delta u_{kji}(t+1) &= \alpha_1\Delta u_{kji}(t) + \eta_1\delta_{kj}x_{k-1,i}^2 \\
        \Delta v_{kji}(t+1) &= \alpha_2\Delta v_{kji}(t) + \eta_2\delta_{kj}x_{k-1,i} 
    \end{align}
    这里 $\alpha_1$ 和 $\alpha_2$ 分别为其动量常数。这个结果与 \cite{MLQP} 一致。

    \section{实现}
    \begin{problem}
        Please implement an on-line BP algorithm for MLQP
(you can use any programming language), train an MLQP
with one hidden layer to classify two spirals problem,
and compare the training time and decision boundaries at
three different learning rates.
    \end{problem}

    \bibliography{ref}
\end{document}